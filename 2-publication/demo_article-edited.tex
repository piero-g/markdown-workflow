\PassOptionsToPackage{unicode=true}{hyperref} % options for packages loaded elsewhere
\PassOptionsToPackage{hyphens}{url}
% custom: no * to svgnames, causes some colors not to be recognized
\PassOptionsToPackage{dvipsnames,svgnames,x11names*}{xcolor}
\documentclass[american,a4paper,]{article}
\usepackage{lmodern}
\usepackage{amssymb,amsmath}
\usepackage{ifxetex,ifluatex}
\usepackage{fixltx2e} % provides \textsubscript
\ifnum 0\ifxetex 1\fi\ifluatex 1\fi=0 % if pdftex
  \usepackage[T1]{fontenc}
  \usepackage[utf8]{inputenc}
  \usepackage{textcomp} % provides euro and other symbols
\else % if luatex or xelatex
  \usepackage{unicode-math}
  \defaultfontfeatures{Ligatures=TeX,Scale=MatchLowercase}
  \newfontfamily{\greekfont}[Script=Greek,Language=Greek]{FreeSerif}
  \newfontfamily{\greekfontsf}[Script=Greek,Language=Greek]{FreeSans}
    \setmainfont[]{Libre Baskerville}
    \setsansfont[]{Libre Franklin}

\fi
% use upquote if available, for straight quotes in verbatim environments
\IfFileExists{upquote.sty}{\usepackage{upquote}}{}
% use microtype if available
\IfFileExists{microtype.sty}{%
\usepackage[]{microtype}
\UseMicrotypeSet[protrusion]{basicmath} % disable protrusion for tt fonts
}{}
\IfFileExists{parskip.sty}{%
\usepackage{parskip}
}{% else
\setlength{\parindent}{0pt}
\setlength{\parskip}{6pt plus 2pt minus 1pt}
}
% custom: footnotes
\usepackage[bottom,hang,multiple]{footmisc}
\setlength{\footnotesep}{4mm}
\usepackage{xcolor}
% custom: url color
\usepackage{hyperref}
\hypersetup{
            pdftitle={Demo Article. An example to test the Pandoc solution},
 % custom
              pdfauthor={John Doe},
              pdfkeywords={Lorem, ipsum, dolor, sit, amet},
            colorlinks=true,
            linkcolor=DarkBlue,
            citecolor=DarkBlue,
            urlcolor=DarkBlue,
            breaklinks=true}
\urlstyle{same}  % don't use monospace font for urls
\usepackage[a4paper,headheight=20pt,bottom=100pt,left=1.25in,right=1.25in]{geometry}
% custom: to improve separations
\widowpenalty=1000
\clubpenalty=1000
% custom: Title Font for headings and titles (\titlefont is set above)
 % custom: sans font for headings
\usepackage{sectsty}
	\allsectionsfont{\sffamily\bfseries\raggedright}
 % custom: for titles
\usepackage{titling}
\renewcommand{\maketitlehooka}{\sffamily}
\usepackage{longtable,booktabs}
% Fix footnotes in tables (requires footnote package)
\IfFileExists{footnote.sty}{\usepackage{footnote}\makesavenoteenv{longtable}}{}
\usepackage{graphicx,grffile}
\makeatletter
\def\maxwidth{\ifdim\Gin@nat@width>\linewidth\linewidth\else\Gin@nat@width\fi}
\def\maxheight{\ifdim\Gin@nat@height>\textheight\textheight\else\Gin@nat@height\fi}
\makeatother
% Scale images if necessary, so that they will not overflow the page
% margins by default, and it is still possible to overwrite the defaults
% using explicit options in \includegraphics[width, height, ...]{}
\setkeys{Gin}{width=\maxwidth,height=\maxheight,keepaspectratio}
\setlength{\emergencystretch}{3em}  % prevent overfull lines
\providecommand{\tightlist}{%
  \setlength{\itemsep}{0pt}\setlength{\parskip}{0pt}}
\setcounter{secnumdepth}{5}
% Redefines (sub)paragraphs to behave more like sections
\ifx\paragraph\undefined\else
\let\oldparagraph\paragraph
\renewcommand{\paragraph}[1]{\oldparagraph{#1}\mbox{}}
\fi
\ifx\subparagraph\undefined\else
\let\oldsubparagraph\subparagraph
\renewcommand{\subparagraph}[1]{\oldsubparagraph{#1}\mbox{}}
\fi

% custom: suppress numbering in figures and tables, caption centered and small
\usepackage[font=small,justification=centering]{caption}
\captionsetup[figure]{labelformat=empty}
\captionsetup[table]{labelformat=empty}

% custom: dates
\usepackage[useregional]{datetime2}

% custom: page number
  \setcounter{page}{1}

% custom: footer and header (except first page)
\usepackage[breakwords]{truncate}
\usepackage{fancyhdr}
\pagestyle{fancy}
\fancyhf{} % clear all default
% left header
\lhead{\small\sffamily Demo Article}
% right reader
\rhead{\small\sffamily Journal Title. Vol. 1 (2018)}
% left footer (DOI)
\lfoot{\small\sffamily \url{https://doi.org/10.5555/12345678}}
% right footer (ID + page (with optional prefix))
\rfoot{\small\sffamily p. draft \thepage}
% custom: TEST for fixing footnotes in first page
\usepackage{etoolbox}

% set default figure placement to htbp
\makeatletter
\def\fps@figure{htbp}
% custom: TEST for fixing footnotes in first page
\patchcmd\maketitle{\renewcommand\thefootnote{\@fnsymbol\c@footnote}}{\AdaptNote\thanks\multthanks}{}{}
\patchcmd\maketitle{%
  \def\@makefnmark{\rlap{\@textsuperscript{\normalfont\@thefnmark}}}}{}{}{}
\makeatother

\ifnum 0\ifxetex 1\fi\ifluatex 1\fi=0 % if pdftex
  \usepackage[shorthands=off,greek,italian,main=american]{babel}
  \newcommand{\textgreek}[2][]{\foreignlanguage{greek}{#2}}
  \newenvironment{greek}[2][]{\begin{otherlanguage}{greek}}{\end{otherlanguage}}
  \newcommand{\textitalian}[2][]{\foreignlanguage{italian}{#2}}
  \newenvironment{italian}[2][]{\begin{otherlanguage}{italian}}{\end{otherlanguage}}
\else
  % load polyglossia as late as possible as it *could* call bidi if RTL lang (e.g. Hebrew or Arabic)
  \usepackage{polyglossia}
  \setmainlanguage[variant=american]{english}
  \setotherlanguage[]{italian}
  \setotherlanguage[]{greek}
\fi

\title{Demo Article. An example to test the Pandoc solution}
 % custom: affiliation only if "affiliationonly" is true, else boxed version below abstract
\author{John Doe}


% custom: dates

 % submitted, revised, accepted, published dates, in YYYY-MM-DD
\date{\small Submitted: \DTMdate{2017-02-05} -- Accepted: \DTMdate{2017-06-18} -- Published: \DTMdate{2018-03-01}}

 % end custom dates

\setlength{\thanksmarkwidth}{1.8em}
\thanksfootextra{}{\hspace{5pt}}



\begin{document}
\maketitle
\fancypagestyle{plain}{%
  \fancyhf{}%
  % right header: journal title (extended) and eISSN
  \rhead{\small\sffamily Journal Title -- A Fictional Scholarly Journal. Vol. 1 (2018) \\ ISSN 1234-567X}
  % left header: section, peer-reviewed plus DOI or journal url
    \lhead{\small\sffamily Essays -- \emph{peer-reviewed} \\ \url{https://doi.org/10.5555/12345678}}
    % right footer (ID + page (with optional prefix))
  	\rfoot{\small\sffamily p. draft \thepage}
      % left footer: copyright block
  \lfoot{\small\sffamily Copyright © 2018 John Doe \\ This work is licensed under the Creative Commons BY License. \\ \url{https://creativecommons.org/licenses/by/4.0/}}
  }

\begin{abstract}
\setlength{\parindent}{0pt}
\par\noindent Lorem ipsum dolor sit amet, \emph{consectetur adipiscing elit}.
Curabitur in ante lobortis, euismod ligula varius, pellentesque ante.
Curabitur suscipit lacus nibh, ut finibus purus -- scelerisque eget.
Vestibulum nec enim odio. Sed feugiat metus iaculis, efficitur massa in,
pulvinar neque. Donec tellus dui, luctus eu efficitur et, tristique a
odio.

Integer faucibus porttitor eros at finibus.
% custom: keywords in abstract
\vspace{1mm}\\ \textbf{Keywords}: Lorem; ipsum; dolor; sit; amet
\end{abstract}
 % custom: acknowledgements
\vspace{1.5em}
\renewcommand{\abstractname}{Acknowledgements}
\begin{abstract}
\setlength{\parindent}{0pt}
\par\noindent Lorem ipsum dolor sit amet, \emph{consectetur adipiscing elit}.
Curabitur in ante lobortis, euismod ligula varius, pellentesque ante.
\end{abstract}

% custom: author blocks
\vspace{2em} % ia safe spacing if lots of authors
\vspace*{\fill} % push to the bottom
\renewenvironment{abstract}
 {\par\noindent\textbf{\abstractname}\ \ignorespaces}
 {\par\medskip}
 % custom: author block with email, orcid and bio
\begin{samepage}
\renewcommand{\abstractname}{John Doe:}
\begin{abstract} % was \sffamily
\small Affiliation (country)
 \\ \url{https://orcid.org/0000-0002-1825-0097}\\ Contact: example@email.org \\ This is a short bio, \emph{italics} and \textbf{bold}. Lorem ipsum dolor
sit amet, consectetur adipiscing elit. Etiam auctor blandit egestas.
Maecenas faucibus eros velit, vitae posuere metus tincidunt et. In vel
justo sed ante iaculis facilisis a quis sapien. Nulla lacinia eu massa
pharetra sollicitudin. Pellentesque tempor elit sit amet libero
fermentum commodo.
\end{abstract}
\end{samepage}

 % if(title)

 % custom: to opt-out to toc in pdf
 % custom: - end if(nopdftoc)
% custom: space from title or separate title page
\newpage
Lorem ipsum dolor sit amet, consectetur adipiscing elit. Fusce sit amet
elit ut enim commodo ultricies a nec sapien. Sed blandit hendrerit
luctus. Lorem ipsum dolor sit amet, consectetur adipiscing elit. Morbi
condimentum non risus nec tempus.

\hypertarget{first-section-with-formatted-text}{%
\section{First section with formatted
text}\label{first-section-with-formatted-text}}

\emph{This sentence is in italic}. Vestibulum ante ipsum primis in
\textbf{a few bold words (strong emphasis)} faucibus orci luctus et
ultrices posuere cubilia Curae; here we have a hard breakline.\\
No new paragraph was created. In auctor pulvinar auctor. Vivamus nulla
lectus, elementum vitae tellus quis, volutpat iaculis tellus.

Here we have a new paragraph. Nulla eget porttitor leo.\footnote{This is
  a simple footnote;} Ut cursus ultrices augue in commodo. Nulla in est
tellus. Nam ac massa at dolor cursus hendrerit quis id orci. Nam nec
erat eget dolor mattis facilisis nec et risus. Aenean mattis erat ac
nisl tincidunt pharetra.\footnote{All external links should be placed in
  footnotes rather than inline, and should be rendered as plain URI,
  such as the following: \url{https://pandoc.org}}

\hypertarget{subsection-with-quotations}{%
\subsection{Subsection with
quotations}\label{subsection-with-quotations}}

Ut vulputate, neque vitae accumsan luctus, risus lorem molestie sapien,
ut vehicula magna diam et dolor.

\begin{quote}
This is a long quotation made with a simple ``Increase left indent.'' In
a libero eu arcu auctor blandit sed quis dui. Nam feugiat ultricies
ligula quis scelerisque. Nulla facilisi. Vivamus rutrum ante eros, ut
elementum est consequat elementum. Vivamus commodo libero facilisis,
suscipit risus vel, egestas felis.
\end{quote}

Integer sollicitudin eleifend neque, vitae faucibus magna consectetur
nec. Vestibulum ante ipsum primis in faucibus orci luctus et ultrices
posuere cubilia Curae; on the other hand ``this is a short quotation.''

\hypertarget{second-section}{%
\section{Second section}\label{second-section}}

\hypertarget{different-languages}{%
\subsection{Different languages}\label{different-languages}}

This is just some dummy text for testing purposes. This is a greek word:
\textgreek{χάρισμα}, while the next quotation is in Italian.

\begin{quote}
\textitalian{Due uomini stavano, l'uno dirimpetto all'altro, al
confluente, per dir così, delle due viottole: un di costoro, a
cavalcioni sul muricciolo basso, con una gamba spenzolata al di fuori, e
l'altro piede posato sul terreno della strada; il compagno, in piedi,
appoggiato al muro, con le braccia incrociate sul petto.}
\end{quote}

\hypertarget{images-and-tables}{%
\subsection{Images and tables}\label{images-and-tables}}

Quisque placerat, diam eget maximus mattis, ipsum risus dignissim nisi,
id rutrum ipsum augue a tortor. Integer quis fringilla odio. Interdum et
malesuada fames ac ante ipsum primis in faucibus.

\begin{figure}
\centering
\includegraphics{demo_article-media/01.jpg}
\caption{Figure 1: a short caption for the image}
\end{figure}

Quisque sit amet ligula ut nulla ultrices dictum. Nulla non hendrerit
erat, nec dictum dolor. Integer quis augue nec neque posuere luctus non
eu massa. Cras diam diam, dictum dignissim turpis et, imperdiet
vestibulum ipsum.

\begin{longtable}[]{@{}lll@{}}
\caption{Simple Table}\tabularnewline
\toprule
& \textbf{Column A} & \textbf{Column B}\tabularnewline
\midrule
\endfirsthead
\toprule
& \textbf{Column A} & \textbf{Column B}\tabularnewline
\midrule
\endhead
\emph{First Row} & Cell 1A & Cell 1B\tabularnewline
\emph{Second Row} & Cell 2A & Cell 2B\tabularnewline
\bottomrule
\end{longtable}

Aliquam eget sapien laoreet velit placerat ornare. Vivamus tempor a eros
id volutpat. Aliquam blandit vitae felis pretium pretium. Nulla lobortis
imperdiet nisi, nec tincidunt neque tempus sed. In hac habitasse platea
dictumst. Cras suscipit nisi vitae cursus semper.

\hypertarget{third-section-messy-formatting}{%
\section{Third Section: messy
formatting}\label{third-section-messy-formatting}}

Different font, different size, different style! Etiam ac fermentum
turpis. Nunc id iaculis ipsum. Aliquam in hendrerit nisi, quis lacinia
lectus. Sed scelerisque interdum neque, ac ultrices nisi interdum sit
amet. Fusce convallis aliquet vehicula.

Various revisions and comments. Pellentesque porta odio porta dui
auctor, a viverra nibh sodales. This is a new revision ante semper ipsum
ullamcorper blandit. Cras iaculis, nisl et convallis blandit, nulla mi
congue urna, et laoreet est nisl quis felis. et metus sed lacinia. Donec
consequat ornare urna sed congue.

Different colors and highlights. Cras ullamcorper, eros nec auctor
mattis, mauris neque sodales velit, et condimentum augue diam eu nunc.
Sed vel malesuada eros. Sed quis ultrices neque, non commodo augue.
Maecenas efficitur sapien eget nisi euismod, eu lacinia leo tempus. Ut
pretium magna ultricies ultrices fermentum. Mauris ultricies
pellentesque consectetur. Proin iaculis mollis dolor, et faucibus
tortor. Lorem ipsum dolor sit amet, consectetur adipiscing elit. Morbi
eget mollis nisl.

\newpage

\hypertarget{references}{%
\section*{References}\label{references}}
\addcontentsline{toc}{section}{References}

Hisakata, R., Nishida, S., \& Johnston, A. (2016). An adaptable metric
shapes perceptual space. \emph{Current Biology}, \emph{26}(14),
1911--1915. \url{https://doi.org/10.1016/j.cub.2016.05.047}

Hogue, C. W. V. (2001). Structure databases. In A. D. Baxevanis \& B. F.
F. Ouellette (Eds.), \emph{Bioinformatics} (2nd ed., pp.~83--109). New
York, NY: Wiley-Interscience.

Musk, E. (2006, August 2). The secret Tesla Motors master plan (just
between you and me). Retrieved September 29, 2016, from
\url{https://www.tesla.com/blog/secret-tesla-motors-master-plan-just-between-you-and-me}

Sambrook, J., \& Russell, D. W. (2001). \emph{Molecular cloning: a
laboratory manual} (3rd ed.). Cold Spring Harbor, NY: CSHL Press

\end{document}
